
\paragraph{Original Plan}
\begin{table}[h]
    \caption{Original Schedule}
    \begin{tabular}{ l c }
    Week started & Task \\
    1-2 & Front-end webapp  \\
    1-3 & NLP CLI  \\
    3-4 & Suggestion Engine 
    \end{tabular} \label{sec:devi-from-init-pln:original-timeline}
\end{table}

The team's first attempt at scheduling out the project was focused on
using agile methodology.  
The overall architecture was broken down into components that could be
designed and the work was scheduled for two week sprints.
This schedule is shown in Table
\ref{sec:devi-from-init-pln:original-timeline}.
The intent was to develop the NLP algorithms, suggestion engine, and
web app in parallel as much as was possible.
The original hope was that the algorithms themselves would be fairly
straightforward to work with and that we would be able to apply
algorithms for reuse easily to quickly develop what we needed.

\paragraph{Challenges}
It became apparent very quickly that this schedule was not going to
work.
The team's experience (summarized in table
\ref{sec:devi-from-init-pln:skillsets} with data mining generally and
machine learning specifically was not up to par.
While the team had some experience in software engineering, the
specific NLP tasks.

\begin{table}[h]
    \caption{Developer Skills}
    \begin{tabular}{ l c l l}
    Name & Data mining & Programming (years) & NLP\\
    Zach DeLong & 1 & 5 & 0 \\
    Monica Metro &  0 & 0 & 0 \\
    Zhangqi Zha & ? & ? & 0\\
    Bikram & ? & ? & ?  
    \end{tabular} \label{sec:devi-from-init-pln:skillsets}
\end{table}

After the project was broken into the above sprints, the team realized
they lacked the expertise to split the work up this way.
Development was started on the web app and on the learning algorithms, and it
became apparent immedaitely that no one understood how to actually
implement LDA or Doc2Vec.

%%% Local Variables:
%%% mode: latex
%%% TeX-master: "../../main"
%%% End:
