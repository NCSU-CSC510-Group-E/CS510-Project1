% We identified the most risky parts as the ones that involved the NLP
% algorithms because we had no idea how to implement them

\paragraph{Prototypes}
% Reference table below.
The prototypes the team identified for this project are listed in
table \ref{sec:planning:prototypes-table}.
% Talk about what the most risky parts of the project are.
The team identified the learning algorithms as the most risky part of
the project, primarily because of the team's lack of experience
with natural language processing algorithms.  

% Go into why these things are risky
As discussed before, these algorithms have succesfully been used to
mine topics for some time, but the particular problem of identifying
the topics in textbooks for search recommendations is fairly novel.
This process requires a data set that matches up well with the desired
% TODO: fill in section
topics, which will come into play in section \ref{sec:eval_plan}.

% mention briefly the problem of evaluating the models created, and
% point ahead to when we'll talk about that more
One other major point of risk is developing the Doc2Vec model.
The Gensim API contains a Doc2Vec model but it does not expose parts
of the API to make the clusters directly available, which is expanded
% TODO: fill this ref in
on in section \ref{??}.

% TODO: What can be expanded on here?

\begin{table*}[t]
  \centering
  \caption{Prototypes and Schedules}
  \begin{tabular}{ l p{10cm} l l}
     \bf Prototype & \bf Requirements & \bf Risk (1-5) & \bf Complete \\ \hline \\
    1 & 
         Implement CLI to make testing of models easier. 
         \newline Implement LDA algorithm using Gensim on static text files.  
         & 3 & + \\ \\
    2 &
        Prototype 1
        Parse Stack Overflow data to text files for input to prototype 1
        Save/Load model so multiple experiments can be run more quickly
        Adapt Prototype 1 to operate on Stack Overflow data.
        & 4 & + \\ \\
    3 &
        Prototype 2
        Prototype database models/access code.
        Prototype similarity measures to be used with learning algorithms
        & 4 & + \\ \\
    4 &
        Implement Doc2Vec
        Implement Doc2Vec with stack overflow data
        & 5 & + \\ \\
    5 &
        Implement code to parse PDFs of textbooks
        Implement connection to library (either REST API or a folder of PDFs
        Apply trained models to textbooks
        Save modeled topics in the database
        & 4 & - \\ \\
    6 &
        Write algorithm to parse saved topics
        Write REST API to serve above algorithm
        & 2 & - \\ \\
    7 &
        Build web-app front end to the above suggestion service
        & 2 & - 
  \end{tabular} \label{sec:planning:prototypes-table}
\end{table*}

\paragraph{Timeline}
The team completely had to re-evalaute the sechedule to accomodate the
prototypes desired in table \ref{sec:planning:prototypes-table}.
The team intended to iterate every week, attempting to finish one
prototype each week on the way around the spiral model.
This often worked, and prototypes 1, 2, and 3 were finished in the
first three weeks of the project.
Prototypes 3 and 4 were develoed in parallel over the last several
weeks of the project and it was followed immedaitely by implementing
the evaluation plan in section \ref{sec:eval_plan}.

%%% Local Variables:
%%% mode: latex
%%% TeX-master: "../../main"
%%% End:
