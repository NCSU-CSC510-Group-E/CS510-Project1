% We identified the most risky parts as the ones that involved the NLP
% algorithms because we had no idea how to implement them

\paragraph{Prototypes}
% Reference table below.
The prototypes the team identified for this project are listed in
table \ref{sec:planning:prototypes-table}.
% Talk about what the most risky parts of the project are.
The team identified the learning algorithms as the most risky part of
the project, primarily because of the team's lack of experience
with natural language processing algorithms.  

% Go into why these things are risky
As discussed before, these algorithms have succesfully been used to
mine topics for some time, but the particular problem of identifying
the topics in textbooks for search recommendations is fairly novel.
This process requires a data set that matches up well with the desired
% TODO: fill in section
topics, which will come into play in section \ref{sec:eval_plan}.

% mention briefly the problem of evaluating the models created, and
% point ahead to when we'll talk about that more
One other major point of risk is developing the Doc2Vec model.
The Gensim API contains a Doc2Vec model but it does not expose parts
of the API to make the clusters directly available, which is expanded
% TODO: fill this ref in
on in section \ref{??}.

% TODO: What can be expanded on here?

\begin{table*}[t]
  \centering
  \caption{Prototypes}
  \begin{tabular}{ l p{10cm} l l}
    % Prototype & Requirements & Risk (1-5) & Complete \\
    1 & 
        \begin{itemize}
        \item Implement CLI to make testing of models easier
        \item Implement LDA algorithm using Gensim on static text files.
        \end{itemize} &
                        3 & + \\
    2 &
        \begin{itemize}
        \item Prototype 1
        \item Parse Stack Overflow data to text files for input to prototype 1
        \item Save/Load model so multiple experiments can be run more quickly
        \item Adapt Prototype 1 to operate on Stack Overflow data.
        \end{itemize} &
                        4 & + \\
    3 &
        \begin{itemize}
        \item Prototype 2
        \item Prototype database models/access code.
        \item Prototype similarity measures to be used with learning
          algorithms
        \end{itemize} &
                        4 & + \\
    4 &
        \begin{itemize}
        \item Implement Doc2Vec
        \item Implement Doc2Vec with stack overflow data
        \end{itemize} &
                        5 & + \\
    5 &
        \begin{itemize}
        \item Implement code to parse PDFs of textbooks
        \item Implement connection to library (either REST API or a folder of PDFs
        \item Apply trained models to textbooks
        \item Save modeled topics in the database
        \end{itemize} &
                        4 & - \\
    6 &
        \begin{itemize}
        \item Write algorithm to parse saved topics
        \item Write REST API to serve above algorithm
        \end{itemize} &
                        2 & - \\
    7 &
        \begin{itemize}
        \item Build web-app front end to the above suggestion service
        \end{itemize} &
                        2 & - 
  \end{tabular} \label{sec:planning:prototypes-table}
\end{table*}

\paragraph{Timeline}
% List timelines for each prototype
% Indicate what got done and what didn't get done

%%% Local Variables:
%%% mode: latex
%%% TeX-master: "../../main"
%%% End:
