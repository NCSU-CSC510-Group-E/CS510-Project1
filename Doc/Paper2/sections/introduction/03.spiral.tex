The solution for our group to our inexperience was to follow a variant
of the spiral model.
The spiral model is a form of risk-driven development where the most
risky parts of a project are identified early and planned into
prototypes that can inform the final product.
% TODO: How did we apply the spiral model?
See the breakdown of the prototypes the team chose in section
\ref{sec:prototypes}.

This afforded us the ability to have regular check-ins and measure our
progress against our goals.
The team convened two to three times a week to do pair programming and
to check progress against the prototypes we proposed.
When combined with regular check-ins with the TAs, we were able to
build prototype programs to perform NLP on a set of training data, to
identify if we could train a set of models to be applied to
textbooks. 

%%% Local Variables:
%%% mode: latex
%%% TeX-master: "../../main"
%%% End:
