% The goal of this sections is to introduce the spiral model.
% Now that we have talked extensively about the problems of the
% project, we need to go into how and why risk driven development was
% the right way to go about solving those problems 

To attack the unforseen complexity of this project, the team
chose the spiral model of software development.
The spiral model is a form of risk-driven development where the most
risky parts of a project are identified early and planned into
prototypes that can inform the final product.
% TODO: How did we apply the spiral model?
See the breakdown of the prototypes the team chose in section
\ref{sec:planning}.

This afforded the team the ability to have regular check-ins and to measure 
progress against the prototypes.
The team convened two to three times a week to do pair programming and
to check progress against the prototypes.
When combined with regular check-ins with the TAs, the team was able to
build prototype programs to perform NLP on a set of training data and
to build various measures of effectiveness of those models.

%%% Local Variables:
%%% mode: latex
%%% TeX-master: "../../main"
%%% End:
