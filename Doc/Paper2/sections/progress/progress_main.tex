% Why did we prioritize our prototypes in the order we did.
% The team thought the NLP algorithms were the riskiest part.
% We wanted to get them out of the way early to make developing the
% web app easier.
% When When working 

The original plan for implementing our system did not adequately
address the issues we had with implementing our learning algorithms,
so the team quickly pivoted to a more risk driven approach as outlined
in section \ref{sec:devi-from-init-pln}.

% What were we able to finish
% We finished prototypes of our natural language processing algorithms
% What actually works so far?  :
The team currently has implemented prototypes 1-4, which means that
the lion's share of NLP processing has been done.
We now have a script that will transform Stack Overflow dumps into
something that our learning algorithms can more easily and efficently
process, as well as a NLP implementations of both LDA and Doc2Vec that
are able to act on the processed data and which implement various
accuracy measures that we have used to detail their usefulness.
We also have save and load functionality worked into our models so
that the models can be reused without the need to train the model on
every run, which has proven to be one of the most useful features
during testing.  


\paragraph{Challenges}
% What challenges did we have on the way?
% NLP algorithms are complciated
% Doc2Vec isn't a topic mining algorithm!  What we are doing with it
% isn't actually natively supported!
% LDA takes FOREVER to run
On the way to getting those prototypes running, we ran into a number
of challenges, chief among them the fact that the team had no
experience in this field, as outlined in seciton \ref{sec:devi-from-init-pln}.
Another major complication we ran into was the simple complexity of
these algorithms.  The Gensim library provides amazingly powerful
implementations for re-use, but that power comes at the cost of
complexity which their reasonably high quality documentation does not
entirely weed out.
The third major stumbling block was the fact that we intended to use
Doc2Vec to mine topics out of text---something it does not natively
do!  
The team came up with a somewhat novel approach to this as outlined in
seciton \ref{sec:validation} after some time,
but this was a major challenge of our research and was met with mixed results.  
The last major challenge was one of time.
Not time to finish the projec, though we could easily have listed
that, but the execution time of the NLP algorithms.
These algorithms are both extremely expensive, and to run them on the
data sets required can take significant portions of time.

% How did we respond to those challenges?
To respond to those challenges, the team spent a lot of time in
research and in documentation.
We spent a lot of time studying the research and trawling through
Gensim documentation
To solve doc2vec we came up with a somewhat novel approach to
modeling topics.
We tried to optomize LDA and implemented caching of models to make
retraining unnecessary once a reasonable model was trained.  

%%% Local Variables:
%%% mode: latex
%%% TeX-master: "../../main"
%%% End:
