% Please do not delete!  thanks! -- zach
% !TEX root = ../main.tex

\subsection{Process Conclusions}
\label{sec:conc:process-conclusions}
% Risk driven development worked fairly well.
This project is the first time anyone in the team applied the
principles of risk-driven development to their work, and the results
were fairly satisfying.
After the initial ambitions gave way to the true complexity of the
project, the team was unsure of how to proceed, and this process
provided a prescription to help dig out of the problem and into a
useful series of prototypes.
The prototypes themselves also served to illuminate the unknown risks
quite well.
For example, it was more or less assumed at the outset of the project
that third party libraries would make the implementation of learning
algorithms trivial and only a few days worth of work, but the process
of implementing and tweaking those algorithms actually consumed the
majority of the team's time.

% We met often enough by the end of the semester, but could have
% started that schedule earlier.
The group's meetings were a somewhat sensetive topic as well.
For the first two weeks of the project, the group met only once a
week, which was not often enough.
After that period, the team met twice a week for at least half an hour
(and most oftem more) to do some pair programming and to brainstorm
our solutions to problems encountered.
This proved to be closer to optimal and significant progress started
being made after that change.

% We used slack to great effect.
The team also made significant use of the chat app Slack.
Since the entire class was already using this tool, the team created a
private channel in the course's Slack server and used that as the
primary touchpoint for communication.
Random questions, brainstorming, research, and general coordination
was quite straightforward in Slack.

% We could have levereged github issues more.
% TODO: does this really add value?  Consider cutting
One more negative organizational conclusion was that the team could
have made significantly better use of an issue-tracking system.
Peer review revealed some platform specific issues and it would have
been easier to coordinate the response to those challenges if there
was one central repository to track progress.


\subsection{LDA}
\label{sec:conc:lda}
% LDA is super dependent on the test set.
As expeted, purely based on the accuracy of the topics, LDA is the
clear winner in our battle of learning algorithms

% LDA takes forever to run!

% LDA can run reasonably well unsupervised.  

\subsection{Doc2Vec}
\label{sec:conc:doc2vec}
% TODO, no idea what to write here yet,



\subsection{Future enhancement}
% TODO
The problem of textbook suggestion is fairly general and lends itself well to a number of different directions.
We would like to start this app off with a fairly limited set of functionality, just the suggestions and a link to the FOSS library holding the book in question, but enhancements such as amazon integration would be 
One major enhancement would be the introduction of a rating system to the suggestion engine.  
Another major enhancement would be including links in the UI to buy the books.
A third major enhancement would be evaluating other topic modeling algorithms to see how they stack up, such as IDA.
Another enhancement would be to test an even more rudimentary topic modeling method, such as simply representing topics as lines in a table of contents.  
Essentially asking the question of whether or not fancy algorithms are
even more useful that simple text search for this domain.

% Can we make a chart of Doc2Vec vs LDA?  Looks like LDA is the winner

%%% Local Variables:
%%% mode: latex
%%% TeX-master: "../../main"
%%% End:
 