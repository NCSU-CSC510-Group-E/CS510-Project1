% Please do not delete!  thanks! -- zach
% !TEX root = ../../main.tex

\section{Introduction}
One of the most basic tasks that a professor must do is identify what textbook to use for a given class.
While some fields have textbooks that have become popular and are clearly the best in their subject matter, many fields (especially emerging ones) have no such exemplar and reading through all possible candidates would be too time consuming to be feasible.  
It would be a complex enough problem if there were not already a massive number of textbooks on sources such as Amazon, but self aware professors have attempted to use textbooks that are less expensive with hopes of allowing lower income students to attend more easily.  
This admirable intention serves to increase the already considerable time needed to find appropriate books, and there is no obvious heuristic to apply to searches to limit the field.  

In the interests of making students lives less expensive, we propose exploring a system which can identify topics in books automatically, and which, given a set of requirements from a user (usually a professor), can suggest options of varying expense and completeness for a given set of topics.  
To do this, we propose exploring word clusters generated by Doc2Vec, a family of common natural language processing (NLP) algorithms that attempt to cluster similar words, and topics generated by latent dirichlet allocation (LDA), another common NLP algorithm that directly attempts to identify topics in text, to automatically infer the content of a set of textbooks.  

We then intend to build a web app that will allow a user to specify a set of topics and which will use the mentioned algorithms to make suggestions while minimizing the cost of said textbooks.  
The actual web app will be a fairly simple single-page web app developed in Angular4 and using Bootstrap to hasten development of a modern UI.  

Initially, the team expects LDA to find more appropriate topics than Doc2Vec simply because it is better purpose built.
That said, Doc2Vec uses a sophisticated neural-network, which may behave better on shorter passages such as a table of contents or an index, allowing us to parse less of the book while still getting useful topics.  